\documentclass{article}
\usepackage[utf8]{inputenc}
\usepackage{graphicx}
\graphicspath{ {images/} }

\title{DSP Project Report: ECG Signal Analysis}
\author{Apoorv Vikram Singh (IMT2013006) \\ Ujjwal Jain (IMT2013056) \\ Abhinav Chawla (IMT2013002) \\ Siddartha Sekhar Padhi (IMT2013043) \\ Rishabh Bhadauria (IMT2013034) \\ Shivam Kumar (IMT2013042) }
\date{\today}

\begin{document}

\maketitle

\section{Introduction}
Electrocardiogram (ECG) is a nearly periodic signal that reflects the activity of the
heart. A lot of information on the normal and pathological physiology of heart can be
obtained from ECG. However, the ECG signals being non-stationary in nature, it is very
difficult to visually analyze them. Thus the need is there for computer based methods for
ECG signal Analysis. \\ \\
We were given the ECG data collected using the Wipro AssureHealth device doing
various physical activities. The sampling frequency was 250Hz. The data given to
us didn't look like an ideal ECG which we generally see in the hospitals. The
sources for additional noise could be powerline interference, electrode contact
noise , the baseline drift and motion artifacts. There could be EMG from the
chest wall. There also could be instrumentation noise. \\ \\
The purpose of this report is to analyse the data and make a C code such that
the output could be given in real-time. 

\section{Data Analysis}
First we analysed the frequency spectrum of the plots given. The signal we
received. The frequency in Hertz was only from 0 to 25 Hz, as can be seen from
the figure. \\
\includegraphics[width=\textwidth]{freq} \\ \\ 
The raw data given to us was told to be baseline filtered signal and the cleaned
signal is the raw signal passed through a bandpass filter. As seen from the
frequency spectrum there is no powerline interference as there is no frequency
of 50Hz and multiple of 50Hz. To remove the powerline interference we can use
a notch filter to remove 50Hz and it's multiple frequencies, that is, 50Hz and
100Hz.\\ \\
However, as per the plot, the signal was not baseline corrected. After
subtracting the signal from it's amplitude mean, we get baseline corrected
signal. This gives us a similar plot as passing the signal from a bandpass
filter, allowing the frequency between 10Hz and 20Hz. \\ \\
The plot here plots the original signal in red, the mean subtraction data is in
blue and the bandpass filter data is in green. \\ 
\includegraphics[width=\textwidth]{baseline} \\ \\ 
For the R-peak detection, after filtering we double differentiated the data, the
basic purpose of which was to amplify the noise. Here the noise in our case can
be thought of as the R-peak. So if there is any peak, it will amplify those.
When subtracting 2 points nearby; if they are close to each other, the points
will suppress, and if they are apart from each-other they will get amplified.
After that we took an optimal window size of 175 samples (found after
exploratory analysis), and found the peaks in the data. \\ \\ 
The original data can be seen and double differentiated data also can be seen
from the plot to clearly see the peaks. \\
\includegraphics[width=\textwidth]{doublediff}
\end{document}
